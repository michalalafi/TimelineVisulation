\pagestyle{empty}   % nestránkováno
\chapter*{A\ \ Testovací arch (přepis)}   % nečíslováno
\thispagestyle{empty}   % nestránkováno
	
	
\paragraph{Úvodní poznámka} \emph{Příloha představuje přepis interaktivního online formuláře, který byl použit pro sběr zpětné vazby od testerů.}

\section*{Úvod}
Dobrý den,\\
předem Vám děkuji za pomoc při testování mojí diplomové práce.
Jejím cílem je vytvořit interaktivní časovou osu, která umožňuje snadné procházení většího množství historických záznamů a~zároveň uživateli dovoluje zobrazovat i~některé vztahy, jež mezi nimi existují.

Než začnete plnit úkoly v~rámci testování, ujistěte se, že máte možnost spustit testovanou aplikaci v~prohlížeči Chrome. Jiné prohlížeče momentálně nejsou podporovány. Pokud jej tedy nemáte nainstalovaný ani instalaci neplánujete, v~testování nepokračujte.

Při plnění úkolů se prosím pokuste potlačit Vaše dějepisné znalosti a~vychá\-zejte pouze z~informací získaných prostřednictvím testované aplikace.

Díky.
\begin{flushright} Michal Kacerovský\end{flushright}
		
\section*{Demografické údaje}
V prvním kroku prosím vyplňte potřebné demografické údaje, které mi umožní snáz porozumět datům získaným během testování.

\vskip\baselineskip
\noindent\emph{Následují formulářové prvky pro získání informací o~věkové kategorii, znalosti klávesových zkratek, znalosti práce s~prohlížečem, zkušenostech se zpracováním dat a~případně jménu a~příjmení.}

\section*{Seznámení s~časovou osou}
Aplikaci naleznete na adrese: \url{http://home.zcu.cz/~kacerov2/timeline}\\
Hlavním předmětem testování je ovládací prvek časové osy v~levé části obrazovky~– tomu věnujte pozornost především. Postranní panel představuje podpůr\-nou aplikaci, do níž je časová osa zasazena a~s~níž také komunikuje. Při testování budete pracovat s~oběma částmi.
\vskip\baselineskip\noindent
Časovou osu lze ovládat myší, klávesnicí i~pomocí tlačítek, která jsou její součástí.
\noindent
Přiblížení/oddálení:
\begin{itemize}
  \setlength\itemsep{0em}
\item[--] posun rolovacího kolečka myši
\item[--] tlačítka + a~– 
\item[--] klávesy + a~– na numerické části klávesnice
\end{itemize}
\noindent
Změna časového rozsahu:
\begin{itemize}
  \setlength\itemsep{0em}
\item[--] klepnutím a~tažením osy levým tlačítkem myši
\item[--] tlačítky vpřed a~vzad
\item[--] kurzorovými klávesami
\end{itemize}
\noindent
Označení záznamu:
\begin{itemize}
  \setlength\itemsep{0em}
\item[--] Klepnutím levého tlačítka myši na záznam
\end{itemize}
(Jakmile v~časové ose označíte některý ze záznamů, v~postranním panelu se zobrazí jeho detailní popis a~v~časové ose pak vztahy označeného záznamu k~jiným nejbližším.)

\noindent
Zrušení označení záznamu:
\begin{itemize}
  \setlength\itemsep{0em}
\item[--] klávesa Escape
\item[--] klepnutí pravým tlačítkem myši kamkoliv mimo záznam
\end{itemize}
\noindent
Vycentrování záznamu:
\begin{itemize}
  \setlength\itemsep{0em}
\item[--] klepnutí pravým tlačítka myši na záznam
\end{itemize}
(Najde takové přiblížení, kdy je záznam největší a~zároveň zobrazen celý. Rovněž jej posune na střed osy.)
\vskip\baselineskip\noindent
Zmíněné interakce si vyzkoušejte a~poté pokračujte dalším krokem.

\section*{Vztahy mezi záznamy}
Časová osa vyvíjená v~rámci této práce se od ostatních liší tím, že zobrazuje mimo samotné záznamy i~vztahy mezi nimi. Ty lze navíc efektivně procházet.
Vztahy jsou v~rámci osy charakterizovány šipkou, přičemž její směr určuje i~význam vztahu. Ten vždy čteme od záznamu, z~něhož šipka vychází, k~záznamu, do něhož směřuje.

\subsubsection*{Příklad}
Michal Kacerovský — potomek $\rightarrow$ Antonín Kacerovský\\
čteme jako \emph{Michal Kacerovský je potomek Antonína Kacerovského},

\noindent
opačně by takový vztah vypadal:\\
Antonín Kacerovský —— potomek $\leftarrow$ Michal Kacerovský

\vskip\baselineskip\noindent
Obdobu můžete pozorovat například u~Břetislava I. a~Spytihněva II. přímo v~ose.
V detailu záznamu pak získáte přehled o~všech vztazích, jichž se účastní, navíc doplněný o~záznamy, které tento vztah rovněž utváří. Klepnutím na jejich název je označíte v~ose a~zobrazíte jejich podrobnosti.
Stejného efektu dosáhnete i~klepnu\-tím na šipku přímo uvnitř osy. Takto můžete například procházet celou pokrevní linii mezi panovníky.

Vyzkoušejte si procházení vztahů a~práci s~nimi.

\section*{Hledání záznamů}
Protože nejsou v~časové ose na první pohled vidět všechny záznamy, můžete využít vyhledávací pole, jež se zobrazuje v~postranním panelu pokaždé, když není označen žádný záznam v~ose.
Zkuste vyhledat třeba Oldřicha.

\section*{Čtení časových údajů}
S použitím aplikace odpovězte na následující otázky.
\begin{enumerate}
  \setlength\itemsep{0em}
\item Kdy vymřeli Přemyslovci po meči?
\item Kdy zemřel Měšek I.?
\item Kdy zemřel bratr Vladislava I.?
\item Kolika let se dožil sourozenec Vratislava I.?
\item Kdy byl korunován Vratislav II. českým králem?
\item V~jakém roce se narodil Oldřichův děd?
\end{enumerate}

\section*{Jaký byl vztah?}
Pomocí časové osy určete vztah mezi následujícími událostmi a~osobnostmi.
\begin{enumerate}
  \setlength\itemsep{0em}
\item Břetislav I. je \underline{(syn/otec/děd/vnuk/strýc/synovec/praděd/pravnuk)} Bole\-slava~II.
\item Boleslav Chrabrý je \underline{(Piastovec/Přemyslovec/Mojmírovec)} .
\item Svatého Václava zavraždil jeho \underline{(otec/syn/děd/bratr/synovec/strýc)}.
\item \underline{(Václav I. / Fridrich II. Štaufský / Fridrich II. Babenberský}\\\underline{Přemysl Otakar I./Přemysl Otakar II.)} obsadil hrad Bítov.
\item Přemysl Otakar II. je \underline{(syn/otec/děd/vnuk/strýc/synovec/praděd/pravnuk)} Přemysla Otakara I.
\end{enumerate}

\section*{Mohli se setkat?}
Pokuste se z~časové osy (bez znalosti konkrétních dat narození a~úmrtí) určit, zda se mohly vybrané dvojice osob setkat (tj. žily ve stejné době).
\begin{enumerate}
\item Břetislav I. a~Boleslav III.
\item Strojmír a~Rostislav
\item Soběslav I. a~první český král
\item Vladivoj a~vrah svatého Václava
\item Vladislav II. a~držitel Zlaté buly sicilské
\item Oldřich a~jeho nejmladší vnuk
\end{enumerate}

\section*{Hodnocení}
Testy jsou za Vámi! Teď je na čase, abyste ohodnotili, jak se Vám s~aplikací pracovalo.
\vskip\baselineskip\noindent
\emph{Následují prvky formuláře pro ohodnocení testovaného widgetu. Posuzované aspekty podrobněji popisuje kapitola \ref{uzivatelske-hodnoceni}.}
\chapter{Úvod}
\label{uvod}
Každý den většina z~nás používá při své práci nebo zábavě internet. Nemusíme pro každou informaci pospíchat do knihovny nebo zevrubně prohledávat tu největší encyklopedii, kterou máme ve své polici, tak jako před několika lety. Ačkoliv v~síti nalezneme neskutečné množství dat a~ve většině případů se i~dopátráme kýženého výsledku, ne vždy můžeme s~určitostí prohlásit, že jde o~správný výsledek, a~ne pokaždé jsme vůbec schopni nalezená data zpracovat.

Obvykle pracujeme s~obyčejnými obrázky, tabulkami a~texty navzájem prová\-zanými pomocí hypertextových odkazů. Co když ale potřebujeme pohlížet na získané informace s~větším odstupem, chceme-li znát spíš vztahy mezi nimi, než jejich obsah jako takový? Pokud budeme zkoumat například rodokmen do druhého kolene, jistě si s~pomocí tužky a~papíru poradíme. Co když ale potřebujeme jít ještě dál a~chceme sledovat vztahy mezi historickými osobami v~řádech stovek, nebo dokonce tisíců let. Na to jen s~naší pamětí a~představivostí nestačíme, obzvlášť máme-li je analyzovat podrobněji.

Předchozí odstavec již ukázal, že příkladem informací, které jsme schopni jen obtížně zpracovávat bez pomoci počítače, jsou právě historické záznamy. Ať už jde o~válečné události, sledování generací královských rodů nebo po minutách monitorovaný teroristický útok, všechna tato data lze uspořádat jak chronologicky v~čase, tak i~v~grafu, jehož hrany představují relace mezi jednotlivými událostmi a~osobami. Ačkoliv člověk zvládne libovolné množství takových záznamů uspořádat v~čase, s~jeho transformací do zmíněného grafu si neporadí už jen proto, že graf takových rozměrů lze jen těžko zpracovat na papíře.

Tato diplomová práce se tedy zaměří na to, jak data historického charakteru vizualizovat, aby z~nich byl snadno patrný jak jejich sled v~čase, tak i~vzájemné vztahy. Zřejmě není možné splnit oba tyto aspekty vizualizace na 100~\%, a~proto tato práce upřednostní chronologické uspořádání před grafovým. Uživatel však bude mít možnost získat informace o~tom, v~jakém vztahu s~okolím je jím zvolený uzel. 

Widget pro HTML stránku, který bude výstupem této práce, pak uživateli umožní přímo v~jeho webovém prohlížeči procházet předzpracovaná data z~externí databáze dostupná prostřednictvím REST rozhraní.
\chapter*{Abstrakt}     % nečíslováno
\thispagestyle{empty}   % nestránkováno
	\section*{Vizuální reprezentace precedenčního grafu}
		Po staletí jsou sbírány a archivovány historické záznamy. Dnes jich tedy existuje velké množství, na které je obtížné získat ucelený pohled. Tato práce se zabývá možnostmi, jakými lze data tohoto charakteru nejen zobrazit ve webovém prohlížeči, ale zároveň i opatřit vizualizací vztahů, jež mezi nimi existují. První část zkoumá historii grafické reprezentace časových dat a upozorňuje na problémy s ní spojené. Na jejich základě stanovuje požadavky pro implementaci ovládacího prvku časové osy a prezentuje některá existující řešení. Poslední kapitoly této práce se zabývají návrhem a realizací vlastní časové osy doplněné o testy a~uživatel\-skou příručku.

\chapter*{Abstract}     % nečíslováno
\thispagestyle{empty}   % nestránkováno
	\section*{Visual representation of precedence graph}
		Over the centuries, historical records are being collected and archived. There is a huge amount of such data, thus it is difficult to get complex insight. This diploma thesis deals with solutions that can be used not only for its chronological displaying in a web browser but even for visualization of relationships that exist in between. The first part examines the history of time data graphical representation and points out related problems. Consequently, it determines requirements for timeline widget implementation and introduces some existent solutions. The last chapters are focused on designing and creating its own timeline widget accompanied by tests and user guide.
